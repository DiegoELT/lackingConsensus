\section{Justification}

Previous works on this topic, such as the ones by Torreno and Perez-Wohlfeil \cite{Tirado2015BreakingTC, PrezWohlfeil2019UltrafastGC}, as well as visualization tools for alignments \cite{Diaz2019MGV} emphasize the importance of HSP identification as a mean to align complete genomes. In addition, plenty of emphasis has been done on the role that \textit{gaps} could take when identifying these.

\medskip

Therefore, we consider necessary to evidence empirically if the use of \textit{gaps} improve (or not) intergenomic alignments. Through it, researchers would have some ease when it comes to defining the genomic distance between two DNA sequences. Furthermore, this would also ease the task of identifying individual mutations that separate two sequences, as well as the evolutive history between them.

\medskip

The paradigm of research for this document is based mainly on previous studies made around the topic of HSP identification in complete genomes. We are able to access open-source software and algorithm implementations which allows us to perform parameterized experiments regarding the use of \textit{gaps}. Visualization tools also allow us to see these alignments in the form of a graphic, which work as a preliminar results to see the quality of an alignment. 

\medskip

Depending on the results obtained from the different algorithms to be compared, both visually and in the metrics we are going to propose, it will be possible to state if the use of \textit{gaps} should be the norm in terms of a consensus intergenomic distance. 