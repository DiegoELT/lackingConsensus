\section{Introduction}

Bioinformatics focuses on the study of biological data through the use of computational tools, such as algorithms and specialized hardware. Technological advances in processes related to Deoxyribonucleic acid (DNA) sequencing have resulted in a considerable impulse to learn its composition, structure, and functions. 

\medskip

The comparison of DNA sequences, which characterize themselves for being a string composed of an alphabet made up by 4 characters (A, C, G, and T), allows us to infer both the structural, as well as the evolutive relationships between individuals of shared or different species \cite{xiong_2006}.

\medskip

However, being able to do this inferences requires substantial computational capacities due to two main reasons. First, the amount of input data that is being sequenced every day through the mentioned advanced techonologies. Second, the development of new algorithms due to the complexity of the problem, which is, in big part, Whole Genome Alignments (WGA).

\medskip

There is a diverse amount of approaches of approaching the WGA problem. In 2015, Torreno \cite{Tirado2015BreakingTC} proposed an out-of-core algorithm to manage the memory constraints associated with it, allowing for similar performance to state-of-the-art techniques for continuous substring identification. These substrings are also called High Scoring Segment Pairs (HSPs). 

\medskip

To accelerate the identification process, Perez-Wohlfeil \cite{PrezWohlfeil2019UltrafastGC} developed an heuristic to remove noise caused by minor alignments, based on the use of unique shared signals between both sequences. Meanwhile, another kind of approach is the one of researchers such as Medvedev \cite{Minkin2019ScalableMW}, who covered algorithms on De Bruijn graphs to find HSPs on closely related species. 

\medskip

A factor present in all the previous mentioned researches is the difficulty to work with sequences which contain interruptions that generate from minor gains or losses of DNA, in the form of small mutations. Said interruptions are often called \textit{gaps} or \textit{indels}, and they are the main focus of this research. 

\medskip

However, when worked with, \textit{gaps} allow to identify larger alighments. Such is the case for Dynamic Programming (DP) algorithms, \textit{e.g.} Needleman-Wunsch and Smith-Waterman, which are staples on comparative genomics. More modern techniques, like Basic Local Alignment Search Tool (BLAST) are even faster, with the downgrade of additional memory requirements. 

\medskip

In comparison, methods that do not employ \textit{gaps} are much more precise, and have fewer memory requirements.

\medskip

The aim of this research document is to further evaluate how intergenomic alignments are affected by the use of \textit{gaps}, and if they should be considered when evaluating the distance between two sequences. Our purpose is to offer an insight on how they affect alignments and which parameters yield better results for WGA as well.

\medskip

The structure of this document is divided into 8 chapters. The second section is going to cover the state of the art for WGA techniques, in a chronological manner. In our third section we explore our theoric framework regarding the different sequence comparison techniques which are relevant to our study. The fourth section goes into deeper analysis of our proposal and how it is going to be developed. The fifth and six sections cover our preliminar results and further experimentation. Finally, we close with a discussion section and with the conclusions that we drew from it.